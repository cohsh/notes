\documentclass[../main]{subfiles}

\begin{document}

\part{導入}
\section{用語}
オブジェクトの集まりを\textbf{集合}と言う。そう命名するとき、個々のオブジェクトは\textbf{元}と呼ばれる。
元の数が有限個である集合は\textbf{有限集合}と呼ばれ、そうでない(?)集合は\textbf{無限集合}と呼ばれる。

\section{記法}
オブジェクト$a$と集合$A$に対して、「$a$が$A$の元である」あるいは「$a$は$A$に属する」を
\[
    a \in A
\]
と書く。そして、その否定を
\[
    a \notin A
\]
と書く。

集合には2つの記法がある。
まず、全ての元をカンマ区切りで書き並べて、中括弧$\{\}$で挟んだ記法を\textbf{外延的記法}と言う(例:$\{0, 1, 2, 3\}$)。全ての元が明らかであるときは、$\cdots$を用いて一部の元を省略した記法も許す。

\end{document}
